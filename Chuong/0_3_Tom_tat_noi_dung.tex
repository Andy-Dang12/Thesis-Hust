\documentclass[../DoAn.tex]{subfiles}
\begin{document}

\begin{center}
    \Large{\textbf{TÓM TẮT NỘI DUNG ĐỒ ÁN}}\\
\end{center}
\vspace{1cm}
Ung thư là một bệnh gây ra bởi những đột biến xảy ra trên DNA, RNA hoặc protein. Những đột biến này có thể dẫn tới sự tăng sinh tế bào một cách bất thường, cơ thể không thể kiểm soát được. Giải trình tự gen giúp phát hiện các đột biến có liên quan đến ung thư, từ đó cho phép đánh giá nguy cơ ung thư hoặc làm tăng hiệu quả của điều trị đối với ung thư. Một bệnh nhân được giải trình tự gene có thể phát hiện được từ 200 - 500 các biến thể hiếm gặp. \cite{1_JC_Taylor} \cite{2_monkol}. Việc điều tra xem một biến thể có phải là nguyên nhân gây bệnh hay không mất khoảng một giờ đồng hồ \cite{3_Dewey}. Vì thế, việc xác định các biến thể  gây bệnh tốn rất nhiều thời gian và công sức. Tuy nhiên ta có thể đẩy nhanh tốc độ chuân đoán nếu bộ gen của bệnh nhân chứa một biến thể gây bệnh đã được nghiên cứu và công bố từ trước đó .Các báo cáo này có thể giải thích một phần hoặc đầy đủ kiểu hình của biến thể đều giúp ích rất nhiều cho việc chẩn đoán.
Chính vì vậy, trong đồ án này, em đã nghiên cứu và thực hiện xây dựng một công cụ tự động thu thập các thông tin về các biến thể từ các bài báo và nghiên cứu khoa học.

\end{document}