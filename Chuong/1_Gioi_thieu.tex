\documentclass[../DoAn.tex]{subfiles}
\begin{document}

\section{Đặt vấn đề}
\label{sec:dvd}
Ung thư là nguyên nhân gây tử vong đứng hàng thứ hai trên toàn cầu và ước tính chiếm 9,6 triệu ca tử vong trong năm 2018. Việt Nam có khoảng 165 000 ca ung thư mới và khoảng 115 000 ca tử vong do ung thư mỗi năm. Ung thư không những làm tổn lại sức khỏe của người bệnh mà còn gây hoang mang cho chính bệnh nhân và người thân, gia đình và cộng đồng. Có nhiều nguyên nhân có thể dẫn đến ung thư, nhưng nguyên nhân chủ yếu là do yếu tố di truyền và do các yếu tố môi trường tác động. Yếu tô di truyền có thể xuất phát từ việc thừa hưởng gene đột biến từ bố , mẹ. Yếu tố môi trường dẫn đến ung thư có thể xảy ra do tiếp xúc với các loại hóa chất gây ung thư, virus, các chất phóng xạ ... Việc xét nghiệm gene di truyền cho phép xác định nguyên nhân gây ung thư là do đột biến di truyền hay do yếu tố môi trường. Bác sĩ có thể căn cứ vào kết quả xét nghiêm để đưa ra các liệu pháp điều trị phù hợp, lên kế hoạch theo dõi các bệnh ung thư thứ phát, và ước lượng nguy cơ mắc bệnh ung thư cho người nhà bệnh nhân. Đối với người chưa bị bệnh, nhưng có tiền sử gia đình về bệnh ung thư di truyền, xét nghiệm gene di truyền sẽ rất có ích cho việc lập kế hoạch theo dõi phát hiện sớm, và thực hiện các biện pháp làm giảm nguy cơ. Tuy nhiên, công việc xét nghiệm và giải trình tự gene là một công việc tốn nhiều chi phí về thời gian và công sức. Vì thế nghiên cứu, ứng dụng những công nghệ mới , với mục tiêu giảm chi phí cho việc xét nghiệm, cũng là giảm chi phí cho những bệnh nhân có nhiều ý nghĩa thực tiễn.

\section{Các giải pháp hiện tại và hạn chế}
\label{sec:giaiphap}
Trước có, cũng đã có những nghiên cứu nhằm phân loại các văn bản y khoa như \cite{useML}. Nghiên cứu này nhằm mục đích phân loại văn bản y khoa theo hai nhiệm vụ khác nhau là phân loại bài báo có gene nhạy cảm với ung thư và những bài báo chứa gene ít nhạy cảm với ung thư. Nhóm tác giả có sử dụng hai mô hình là SVM và CNN, theo đánh giá cá nhân, thì 2 mô hình này có tốc độ huấn luyện và tốc độ dự đoán tương đối chậm. Nguyên nhân là do độ phức tạp tính toán của SVM tăng khá nhanh theo số lượng điểm dữ liệu và số chiều của dữ liệu. Trong khi đó, các bài toán về ngôn ngữ đều có kích thước của vector đặc trưng là tương đối lớn, nhất là khi tác giả còn sử dụng phương pháp trích chọn đặc trưng là "bag of n gram" (về cơ bản, phương pháp này tạo từ điển cho từng n từ. Việc này còn khiến cho vector đặc trưng còn tăng kích thước lên nhiều lần). Bên cạnh đó mô hình SVM, tác giả còn sửa dụng mô hình CNN với phép tính tích chập (convolution) có kích thước 3x1, mô hình CNN có chi phí tính toán cho quá trình huấn luyện (training) và dự đoán (inference) khá lớn, và thông thường những mô hình convolutional neural network đều cố gắng tận dụng khả năng tính toán song song của GPU để giảm thời gian chạy.  Mặc dù có những nhược điểm kể trên nhưng kết quả mà nhóm tác giả đạt được là rất hứa hẹn khi lượng dữ liệu chỉ khoảng 4000 văn bản được gán nhãn thủ công, độ chính xác (accuracy) cũng đã lên đến 0.8852 . Ngoài ra còn có \cite{GAPscreener} đạt độ chính xác cũng khá cao khi sử dụng mô hình SVM, tuy nhiên tốc độ chưa thật sự ấn tượng. So với những việc mà \cite{useML} đã thực viện thì nhóm tác giả của \cite{GAPscreener} đã tiến hành phân loại văn bản y khoa theo một hướng có vẻ dễ dàng hơn là phân loại các bài báo thành 2 nhóm là có liên quan đến ung thư và không liên quan đến ung thư. Tác giả cũng sử dụng bộ dữ liệu thu thập tự động những khá tin cậy. Bộ dữ liệu này bao gồm 20.000 văn bản toàn văn. Mỗi nhóm gồm 10.000 dữ liệu. Tác giả vẫn sử dụng mô hình quen thuộc là SVM với hạt nhâm (kernel) sử dụng là RBF. Độ chính xác của mô hình đạt khá tốt khi nó lên đến 0.975 .Tốc độ phân loại mà SVM mang lại là 0.02s / 1 văn bản. Độ chính xác của mô hình SVM là khá tốt , tuy nhiên thời gian chạy phân loại thuật toán là chưa tốt. 

\section{Mục tiêu và định hướng giải pháp}
Trước những bài toán thực tế nêu trên, trong đồ án này, em đã cố gắng tổng hợp những nghiên cứu trước đó. Ngoài ra để giải quyết vấn đề về phân loại bài toán y khoa, đồng thời tổng hợp những thông tin về mô tả về gene bệnh trong các các bài báo về y khoa đó, em đã đề xuất xây dựng một công cụ gồm 3 thành phần: một là công cụ thu thập dữ liệu về các thông tin liên quan đến bài báo như: PIMD, tiêu đề, tóm tắt, chỉ số DOI, các bài báo trích dẫn, các bài báo tham khảo, các bài báo được Pubmed gợi ý là tương tự. Mỗi thông tin trên đều có vài trò riêng trong việc thu thập văn bản toàn văn về bệnh ung thư. Hai là công cụ sàng lọc, loại bỏ những bài báo không liên quan đến ung thư dựa với cách tiếp cận theo hướng xử lý ngôn ngữ tự nhiên, thuật toán áp dụng là các thuật toán học máy. Ba là đưa ra các biểu thức chính quy để tìm kiếm các mô tả biến thể được đề cập trong các bài báo đã được xác định từ trước là có liên quan đến ung thư.

\section{Đóng góp của đồ án}
Trong phạm vi đồ án này, em đã xây dựng một công cụ tự động thu thập dữ liệu y tế từ cớ sở pubmed. Công cụ này có thể chạy hoàn toàn tự động do nó có khả năng tìm kiếm nhiều bài báo chỉ từ một bài báo cho trước. Việc này sẽ làm giảm đáng kể sức người khi nó hạn chế được việc phải thu thập rất nhiều PMID để tìm kiếm. Điều này trước đây chưa có ai thực hiện. Thứ hai, em đã xây dựng được một công cụ sàng lọc văn bản dựa trên mô hình Naive Bayes classifier, công cụ này có chỉ số recall trung bình là 0.9821. Thứ 3, áp dụng 47 biểu thức chính quy để tìm kiếm các biến thể đột biến được đề cập trong bài báo. Việc này giúp cho việc tìm kiếm biến thể trở nên cực kỳ nhanh chóng, khi nó có thể tìm thấy hơn 324.000 biến thể chỉ trong thời gian 52 phút trên CPU chỉ chạy 1 luồng duy nhất.


\section{Bố cục đồ án}
Đồ án được chia thành 4 chương Trong đó chương 2 trình bày về đánh giá sơ bộ 2 bài báo có liên quan và trình bày cơ sở lý thuyết của những thuật toán học máy như các phương pháp trích chọn đặc trưng, cách đánh giá mô hình phân loại, và một số thuật toán học máy điển hình đã được áp dụng để triển khai bài toán. Chương 3 em đưa ra đề xuất để giải quyết bài toán này, bao gồm cách thu thập dữ liệu từ 3 cơ sở dữ liệu khác nhau là HGMD, Pubmed và Pubmed Central, cũng như huấn luyện mô hình học máy cho bài toán phân loại văn bản y khoa, triển khai các biểu thức chính quy để tìm kiếm biến thể. Chương 4 em trình bày về những kết quả thực tế khi triển khai bài toán, và đánh giá kết quả đã đạt được.  Và phần cuối cùng là phần 5. Phần này em đã tổng kết lại những kết quả đã đạt được khi thực hiện đồ án. Tuy nhiên đồ án này vẫn chưa hoàn thiện, em cũng đã chỉ ra những thiếu sót của bản thân và hướng phát triển để có thể đưa ứng dụng này vào triển khai trong thực tế.

\end{document}