\documentclass[../DoAn.tex]{subfiles}
\begin{document}

\section{Kết luận}

Đồ án đã thực hiện được các mục tiêu cơ bản khi thực hiện:
\begin{itemize}
 \item Tìm hiểu vai trò của giải trình tự gene trong chẩn đoán ung thư.
 \item Xây dựng công cụ tự động thu thập dũ liệu về các bài báo khoa học có liên quan đến bệnh ung thư từ cơ sở dữ liệu pubmed.
 \item Thực hiện thu thập dữ liệu, xây dựng mô hình phân loại văn bản dựa trên các thuật toán học máy truyền thống.
 \item Tìm kiếm các mô tả biến thể được đề cập trong bài báo khoa học.
\end{itemize}

\section{Hướng phát triển trong tương lai}

Do còn kiến thức về y sinh còn hạn chế, nên công cụ vẫn còn đơn giản. Những biến thể được tìm thấy đều là những bằng chứng chưa được xác nhận bởi các chuyên gia. Vì thế, để phát triển thêm và đưa vào ứng dụng trong thực tế, công cụ cần câng cấp thêm tính năng cao cấp hơn như:
\begin{itemize}
 \item Bên cạnh việc tìm kiếm các biến thể / đột biến, công cụ cũng cần trích xuất tên loại gene được đề cập trong bài báo.
 \item Ghép nối loại gene và đột biến đã tìm thấy để chỉ ra được: đột biến này xảy ra trên đoạn gene nào.
 \item Xây dựng cơ sở dữ liệu để lưu trữ các đột biến tìm thấy.
 \item Xây dựng giao diện để các bác sĩ có thể thao tác thuận tiện.
 \item Cơ sở dữ liệu pubmed liệt kê rất đầy đủ các trích dẫn của bài báo khoa học. Ta có thể sử dụng dữ liệu trích dẫn này ở dạng đồ thị và xây dựng mô hình phân loại đỉnh (node classify) thay thế cho mô hình phân loại văn bản.
\end{itemize}

\end{document}